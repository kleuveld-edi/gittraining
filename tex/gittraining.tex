%%%%%%%%%%%%%%%%%%%%%%%%%%%%%%%%%%%%%%%%%
% Beamer Presentation
% LaTeX Template
% Version 1.0 (10/11/12)
%
% This template has been downloaded from:
% http://www.LaTeXTemplates.com
%
% License:
% CC BY-NC-SA 3.0 (http://creativecommons.org/licenses/by-nc-sa/3.0/)
%
%%%%%%%%%%%%%%%%%%%%%%%%%%%%%%%%%%%%%%%%%

%----------------------------------------------------------------------------------------
%	PACKAGES AND THEMES
%----------------------------------------------------------------------------------------

\documentclass{beamer}
\usepackage{xcolor}
\mode<presentation> {

% The Beamer class comes with a number of default slide themes
% which change the colors and layouts of slides. Below this is a list
% of all the themes, uncomment each in turn to see what they look like.

%\usetheme{default}
%\usetheme{AnnArbor}
%\usetheme{Antibes}
%\usetheme{Bergen}
%\usetheme{Berkeley}
%\usetheme{Berlin}
%\usetheme{Boadilla}
%\usetheme{CambridgeUS}
%\usetheme{Copenhagen}
%\usetheme{Darmstadt}
%\usetheme{Dresden}
%\usetheme{Frankfurt}
%\usetheme{Goettingen}
\usetheme{Hannover}
%\usetheme{Ilmenau}
%\usetheme{JuanLesPins}
%\usetheme{Luebeck}
%\usetheme{Madrid}
%\usetheme{Malmoe}
%\usetheme{Marburg}
%\usetheme{Montpellier}
%\usetheme{PaloAlto}
%\usetheme{Pittsburgh}
%\usetheme{Rochester}
%\usetheme{Singapore}
%\usetheme{Szeged}
%\usetheme{Warsaw}

% As well as themes, the Beamer class has a number of color themes
% for any slide theme. Uncomment each of these in turn to see how it
% changes the colors of your current slide theme.

%\usecolortheme{albatross}
%\usecolortheme{beaver}
%\usecolortheme{beetle}
%\usecolortheme{crane}
%\usecolortheme{dolphin}
%\usecolortheme{dove}
%\usecolortheme{fly}
%\usecolortheme{lily}
%\usecolortheme{orchid}
%\usecolortheme{rose}
%\usecolortheme{seagull}
\usecolortheme{seahorse}
%\usecolortheme{whale}
%\usecolortheme{wolverine}

%\setbeamertemplate{footline} % To remove the footer line in all slides uncomment this line
%\setbeamertemplate{footline}[page number] % To replace the footer line in all slides with a simple slide count uncomment this line

%\setbeamertemplate{navigation symbols}{} % To remove the navigation symbols from the bottom of all slides uncomment this line
}

\usepackage{graphicx} % Allows including images
\usepackage{booktabs} % Allows the use of \toprule, \midrule and \bottomrule in tables

%----------------------------------------------------------------------------------------
%	TITLE PAGE
%----------------------------------------------------------------------------------------

\title[Git]{Introduction to GitHub} % The short title appears at the bottom of every slide, the full title is only on the title page

\author{Koen Leuveld} % Your name
\institute[EDI] % Your institution as it will appear on the bottom of every slide, may be shorthand to save space
{
EDI \\ % Your institution for the title page
\medskip
\textit{k.leuveld@surveybe.com} % Your email address
}
\date{\today} % Date, can be changed to a custom date

\begin{document}

\begin{frame}
\titlepage % Print the title page as the first slide
\end{frame}

\begin{frame}
\frametitle{Overview} % Table of contents slide, comment this block out to remove it
\tableofcontents % Throughout your presentation, if you choose to use \section{} and \subsection{} commands, these will automatically be printed on this slide as an overview of your presentation
\end{frame}

%----------------------------------------------------------------------------------------
%	PRESENTATION SLIDES
%----------------------------------------------------------------------------------------

%------------------------------------------------
\section{Introduction} % Sections can be created in order to organize your presentation into discrete blocks, all sections and subsections are automatically printed in the table of contents as an overview of the talk
%------------------------------------------------

%\subsection{Subsection Example} % A subsection can be created just before a set of slides with a common theme to further break down your presentation into chunks

\begin{frame}
\frametitle{What is git?}
\begin{itemize}
\item Git is a Version Control System
\item It was developed by the creator of Linux to keep track of the work of all the people helping out on developing Linux.
\item It allows you to download the project you’re working on from a central server, make all the edits you want, and upload those back to the server.
\item It works as a set of arcane commands you type from the command line. But there are programs that make it easier, like GitHub Desktop.
\end{itemize}
\end{frame}

%------------------------------------------------

\begin{frame}
	\frametitle{Why Git?}
	\begin{itemize}
		\item It handles conflicts really nicely (in plain text files, like CSV, .do and .ado).
		\item It allows you to revert any change (any commit, in git-speak)
		\item It allows you to download any old version of any file. (Though for this you will need to use to the command line.)
	\end{itemize}
\end{frame}


\begin{frame}
\frametitle{Setting up Git}
	\begin{itemize}
		\item Download and install GitHub Desktop: https://desktop.github.com/
		\item Optional, but recommended: Download Git: https://git-scm.com/downloads 
		\item Add the training repository to your GitHub Desktop: https://github.com/kleuveld-edi/gittraining.git
	\end{itemize}
\end{frame}

\section{Using Git} % Sections can be created in order to organize your presentation into discrete blocks, all sections and subsections are automatically printed in the table of contents as an overview of the talk
\subsection{Basic Use}
\begin{frame}
	\frametitle{Git workflow}
	\only<1>{The general workflow to get your changes onto the git server is:}
	\only<2>{Expanding on this:}
	\only<3,4>{So forget about staging!}
	\only<5>{All things together:}
	\begin{enumerate}
  		\item<1-> Change or add File
  		\only<1-2>{\item \textbf{Stage} changed file} \only<3->{\item \textcolor{gray}{\textbf{Stage} changed file}}
  		\only<2>{
  			\begin{itemize}
	  			\item This tells Git that you've changed the file. 
	  			\item GitHub Desktop does this automatically!
  			\end{itemize}
  		}
	  	\item<1-> \textbf{Commit} change
		\only<3,5>{
			\begin{itemize}
				\item This adds the changes you've staged to your \textbf{local} repository.
				\item Git now saves a snapshot of the state of your project.
				\item This means you can undo all the changes you've made since the last commit.
			\end{itemize}
		}
	  	\item<1-> \textbf{Push} Commit(s)
	   	\only<4,5>{
	   		\begin{itemize}
	  			\item Commits will be uploaded to the remote repository (the github server for now).
				\item Others can now \textbf{fetch} your changes
	  		\end{itemize}
	  	}
	\end{enumerate}
\end{frame}


\begin{frame}
	\frametitle{Now you!}
	\begin{enumerate}
		\item Add a file to the folder where you've put your repository. For example: Koen.txt
		\item Commit the change (press "Commit to Master" in bottom left). GitHub Desktop may prefill a summary for you. 
		  		These are always in present tense, as it describes what is done when the commit is applied.
		\item Push the change to the central repository (press "Push origin" in the top menu).
		\item The "Push Origin" button may change to "Fetch changes" as the commits of other people are applied. 
		  		Press the button again to download them
	\end{enumerate}
\end{frame}

\begin{frame}
	\frametitle{Notes:}
	\begin{enumerate}
		\item It all works fairly similar to Dropbox, but you have to do things manually. This may or may not be a good thing.
		\item You can do all we've done from the command line, but it'd be more trouble than it's worth.
		\item In the "History" tab of GitHub Desktop you can see all the commits. You can right-click any commit to revert it. 
		\item Reverting creates a new commit, which you can then revert as well!
	\end{enumerate}
\end{frame}

\subsection{Advanced Use}
\begin{frame}
	\frametitle{Resolving Conflicts}
	...
\end{frame}


\begin{frame}
	\frametitle{Viewing the History of a file}
	Use the github website. Note how much information there is to process, not quite user-friendly!
\end{frame}

\begin{frame}
	\frametitle{Branches}
	...
\end{frame}

% %------------------------------------------------

% \begin{frame}
% \frametitle{Blocks of Highlighted Text}
% \begin{block}{Block 1}
% Lorem ipsum dolor sit amet, consectetur adipiscing elit. Integer lectus nisl, ultricies in feugiat rutrum, porttitor sit amet augue. Aliquam ut tortor mauris. Sed volutpat ante purus, quis accumsan dolor.
% \end{block}

% \begin{block}{Block 2}
% Pellentesque sed tellus purus. Class aptent taciti sociosqu ad litora torquent per conubia nostra, per inceptos himenaeos. Vestibulum quis magna at risus dictum tempor eu vitae velit.
% \end{block}

% \begin{block}{Block 3}
% Suspendisse tincidunt sagittis gravida. Curabitur condimentum, enim sed venenatis rutrum, ipsum neque consectetur orci, sed blandit justo nisi ac lacus.
% \end{block}
% \end{frame}

% %------------------------------------------------

% \begin{frame}
% \frametitle{Multiple Columns}
% \begin{columns}[c] % The "c" option specifies centered vertical alignment while the "t" option is used for top vertical alignment

% \column{.45\textwidth} % Left column and width
% \textbf{Heading}
% \begin{enumerate}
% \item Statement
% \item Explanation
% \item Example
% \end{enumerate}

% \column{.5\textwidth} % Right column and width
% Lorem ipsum dolor sit amet, consectetur adipiscing elit. Integer lectus nisl, ultricies in feugiat rutrum, porttitor sit amet augue. Aliquam ut tortor mauris. Sed volutpat ante purus, quis accumsan dolor.

% \end{columns}
% \end{frame}

% %------------------------------------------------
% \section{Second Section}
% %------------------------------------------------

% \begin{frame}
% \frametitle{Table}
% \begin{table}
% \begin{tabular}{l l l}
% \toprule
% \textbf{Treatments} & \textbf{Response 1} & \textbf{Response 2}\\
% \midrule
% Treatment 1 & 0.0003262 & 0.562 \\
% Treatment 2 & 0.0015681 & 0.910 \\
% Treatment 3 & 0.0009271 & 0.296 \\
% \bottomrule
% \end{tabular}
% \caption{Table caption}
% \end{table}
% \end{frame}

% %------------------------------------------------

% \begin{frame}
% \frametitle{Theorem}
% \begin{theorem}[Mass--energy equivalence]
% $E = mc^2$
% \end{theorem}
% \end{frame}

% %------------------------------------------------

% \begin{frame}[fragile] % Need to use the fragile option when verbatim is used in the slide
% \frametitle{Verbatim}
% \begin{example}[Theorem Slide Code]
% \begin{verbatim}
% \begin{frame}
% \frametitle{Theorem}
% \begin{theorem}[Mass--energy equivalence]
% $E = mc^2$
% \end{theorem}
% \end{frame}\end{verbatim}
% \end{example}
% \end{frame}

% %------------------------------------------------

% \begin{frame}
% \frametitle{Figure}
% Uncomment the code on this slide to include your own image from the same directory as the template .TeX file.
% %\begin{figure}
% %\includegraphics[width=0.8\linewidth]{test}
% %\end{figure}
% \end{frame}

% %------------------------------------------------

% \begin{frame}[fragile] % Need to use the fragile option when verbatim is used in the slide
% \frametitle{Citation}
% An example of the \verb|\cite| command to cite within the presentation:\\~

% This statement requires citation \cite{p1}.
% \end{frame}

% %------------------------------------------------

% \begin{frame}
% \frametitle{References}
% \footnotesize{
% \begin{thebibliography}{99} % Beamer does not support BibTeX so references must be inserted manually as below
% \bibitem[Smith, 2012]{p1} John Smith (2012)
% \newblock Title of the publication
% \newblock \emph{Journal Name} 12(3), 45 -- 678.
% \end{thebibliography}
% }
% \end{frame}

% %------------------------------------------------

% \begin{frame}
% \Huge{\centerline{The End}}
% \end{frame}

% %----------------------------------------------------------------------------------------

\end{document} 